\documentclass[]{elsarticle} %review=doublespace preprint=single 5p=2 column
%%% Begin My package additions %%%%%%%%%%%%%%%%%%%
\usepackage[hyphens]{url}
\usepackage{lineno} % add
\providecommand{\tightlist}{%
  \setlength{\itemsep}{0pt}\setlength{\parskip}{0pt}}

\bibliographystyle{elsarticle-harv}
\biboptions{sort&compress} % For natbib
\usepackage{graphicx}
\usepackage{booktabs} % book-quality tables
%% Redefines the elsarticle footer
%\makeatletter
%\def\ps@pprintTitle{%
% \let\@oddhead\@empty
% \let\@evenhead\@empty
% \def\@oddfoot{\it \hfill\today}%
% \let\@evenfoot\@oddfoot}
%\makeatother

% A modified page layout
\textwidth 6.75in
\oddsidemargin -0.15in
\evensidemargin -0.15in
\textheight 9in
\topmargin -0.5in
%%%%%%%%%%%%%%%% end my additions to header

\usepackage[T1]{fontenc}
\usepackage{lmodern}
\usepackage{amssymb,amsmath}
\usepackage{ifxetex,ifluatex}
\usepackage{fixltx2e} % provides \textsubscript
% use upquote if available, for straight quotes in verbatim environments
\IfFileExists{upquote.sty}{\usepackage{upquote}}{}
\ifnum 0\ifxetex 1\fi\ifluatex 1\fi=0 % if pdftex
  \usepackage[utf8]{inputenc}
\else % if luatex or xelatex
  \usepackage{fontspec}
  \ifxetex
    \usepackage{xltxtra,xunicode}
  \fi
  \defaultfontfeatures{Mapping=tex-text,Scale=MatchLowercase}
  \newcommand{\euro}{€}
\fi
% use microtype if available
\IfFileExists{microtype.sty}{\usepackage{microtype}}{}
\ifxetex
  \usepackage[setpagesize=false, % page size defined by xetex
              unicode=false, % unicode breaks when used with xetex
              xetex]{hyperref}
\else
  \usepackage[unicode=true]{hyperref}
\fi
\hypersetup{breaklinks=true,
            bookmarks=true,
            pdfauthor={},
            pdftitle={Regional Liquor Sales in Iowa},
            colorlinks=true,
            urlcolor=blue,
            linkcolor=magenta,
            pdfborder={0 0 0}}
\urlstyle{same}  % don't use monospace font for urls
\setlength{\parindent}{0pt}
\setlength{\parskip}{6pt plus 2pt minus 1pt}
\setlength{\emergencystretch}{3em}  % prevent overfull lines
\setcounter{secnumdepth}{0}
% Pandoc toggle for numbering sections (defaults to be off)
\setcounter{secnumdepth}{0}
% Pandoc header


\usepackage[nomarkers]{endfloat}

\begin{document}
\begin{frontmatter}

  \title{Regional Liquor Sales in Iowa}
    \author[CUNY School of Professional Studies]{Christophe Hunt\corref{c1}}
   \ead{christophe.hunt@spsmail.cuny.edu} 
   \cortext[c1]{Author}
    \author[CUNY School of Professional Studies]{Senthil Dhanapal}
   \ead{senthil.dhanapal@spsmail.cuny.edu} 
  
    \author[CUNY School of Professional Studies]{Yadu Chittampalli}
   \ead{yadu.chittampalli@spsmail.cuny.edu} 
  
      \address[CUNY School of Professional Studies]{CUNY School of Professional Studies, Data Analytics, New York, NY}
  
  \begin{abstract}
  This is the abstract.
  
  It consists of two paragraphs
  
  \section{\texorpdfstring{Keywords: \emph{Liquor, Liquor
  Sales.}}{Keywords: Liquor, Liquor Sales.}}\label{keywords-liquor-liquor-sales.}
  \end{abstract}
  
 \end{frontmatter}

\section{Problem}\label{problem}

Liquor sales are highly variable and the objective of this report is to
create a statiscal model for the number of bottle sold by regional. This
will help us predict inventory and assist wholesale distributors and the
State of Iowa adjust inventory projections accordingly.

\section{Introduction}\label{introduction}

describe the background and motivation of your problem.

\section{Research Background (Literature
Review)}\label{research-background-literature-review}

\section{Methodology}\label{methodology}

discuss the key aspects of your problem, data set and regression
model(s). Given that you are working on real-world data, explain at a
high-level your exploratory data analysis, how you prepared the data for
regression modeling, your process for building regression models, and
your model selection.

\section{Experimentation and Results}\label{experimentation-and-results}

describe the specifics of what you did (data exploration, data
preparation, model building, model selection, model evaluation, etc.),
and what you found out (statistical analyses, interpretation and
discussion of the results, etc.).

\section{Discussion and Conclusions}\label{discussion-and-conclusions}

conclude your findings, limitations, and suggest areas for future work.

\section*{References}\label{references}
\addcontentsline{toc}{section}{References}

\section{Appendices}\label{appendices}

\subsection{Supplemental tables and/or
figures.}\label{supplemental-tables-andor-figures.}

\section{Session Info}\label{session-info}

\begin{itemize}\raggedright
  \item R version 3.3.2 (2016-10-31), \verb|x86_64-w64-mingw32|
  \item Locale: \verb|LC_COLLATE=English_United States.1252|, \verb|LC_CTYPE=English_United States.1252|, \verb|LC_MONETARY=English_United States.1252|, \verb|LC_NUMERIC=C|, \verb|LC_TIME=English_United States.1252|
  \item Base packages: base, datasets, graphics, grDevices,
    methods, stats, utils
  \item Loaded via a namespace (and not attached):
    backports~1.0.4, digest~0.6.10, evaluate~0.10,
    htmltools~0.3.5, knitr~1.15.1, magrittr~1.5, Rcpp~0.12.8,
    rmarkdown~1.2, rprojroot~1.1, rticles~0.2, stringi~1.1.2,
    stringr~1.1.0, tools~3.3.2, yaml~2.1.14
\end{itemize}

\subsection{R statistical programming
code.}\label{r-statistical-programming-code.}

\subsection{R source code}\label{r-source-code}

Please see
\href{https://github.com/ChristopheHunt/DATA-621-Group-1/blob/master/Homework\%205/Homework\%205.Rmd}{Final
Project.rmd} on GitHub for source code.

https://github.com/ChristopheHunt/DATA-621-Group-1/blob/master/Homework\%205/Homework\%205.Rmd

\end{document}


