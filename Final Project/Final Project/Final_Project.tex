\documentclass[]{elsarticle} %review=doublespace preprint=single 5p=2 column
%%% Begin My package additions %%%%%%%%%%%%%%%%%%%
\usepackage[hyphens]{url}
\usepackage{lineno} % add
\providecommand{\tightlist}{%
  \setlength{\itemsep}{0pt}\setlength{\parskip}{0pt}}

\bibliographystyle{elsarticle-harv}
\biboptions{sort&compress} % For natbib
\usepackage{graphicx}
\usepackage{booktabs} % book-quality tables
%% Redefines the elsarticle footer
%\makeatletter
%\def\ps@pprintTitle{%
% \let\@oddhead\@empty
% \let\@evenhead\@empty
% \def\@oddfoot{\it \hfill\today}%
% \let\@evenfoot\@oddfoot}
%\makeatother

% A modified page layout
\textwidth 6.75in
\oddsidemargin -0.15in
\evensidemargin -0.15in
\textheight 9in
\topmargin -0.5in
%%%%%%%%%%%%%%%% end my additions to header

\usepackage[T1]{fontenc}
\usepackage{lmodern}
\usepackage{amssymb,amsmath}
\usepackage{ifxetex,ifluatex}
\usepackage{fixltx2e} % provides \textsubscript
% use upquote if available, for straight quotes in verbatim environments
\IfFileExists{upquote.sty}{\usepackage{upquote}}{}
\ifnum 0\ifxetex 1\fi\ifluatex 1\fi=0 % if pdftex
  \usepackage[utf8]{inputenc}
\else % if luatex or xelatex
  \usepackage{fontspec}
  \ifxetex
    \usepackage{xltxtra,xunicode}
  \fi
  \defaultfontfeatures{Mapping=tex-text,Scale=MatchLowercase}
  \newcommand{\euro}{€}
\fi
% use microtype if available
\IfFileExists{microtype.sty}{\usepackage{microtype}}{}
\ifxetex
  \usepackage[setpagesize=false, % page size defined by xetex
              unicode=false, % unicode breaks when used with xetex
              xetex]{hyperref}
\else
  \usepackage[unicode=true]{hyperref}
\fi
\hypersetup{breaklinks=true,
            bookmarks=true,
            pdfauthor={},
            pdftitle={Regional Liquor Sales in Iowa},
            colorlinks=true,
            urlcolor=blue,
            linkcolor=magenta,
            pdfborder={0 0 0}}
\urlstyle{same}  % don't use monospace font for urls
\setlength{\parindent}{0pt}
\setlength{\parskip}{6pt plus 2pt minus 1pt}
\setlength{\emergencystretch}{3em}  % prevent overfull lines
\setcounter{secnumdepth}{0}
% Pandoc toggle for numbering sections (defaults to be off)
\setcounter{secnumdepth}{0}
% Pandoc header


\usepackage[nomarkers]{endfloat}

\begin{document}
\begin{frontmatter}

  \title{Regional Liquor Sales in Iowa}
    \author[CUNY School of Professional Studies]{Christophe Hunt\corref{c1}}
   \ead{christophe.hunt@spsmail.cuny.edu} 
   \cortext[c1]{Author}
    \author[CUNY School of Professional Studies]{Senthil Dhanapal}
   \ead{senthil.dhanapal@spsmail.cuny.edu} 
  
    \author[CUNY School of Professional Studies]{Yadu Chittampalli}
   \ead{yadu.chittampalli@spsmail.cuny.edu} 
  
      \address[CUNY School of Professional Studies]{CUNY School of Professional Studies, Data Analytics, New York, NY}
  
  \begin{abstract}
  This is the abstract.
  
  It consists of two paragraphs
  
  \section{\texorpdfstring{Keywords: \emph{Liquor, Liquor
  Sales.}}{Keywords: Liquor, Liquor Sales.}}\label{keywords-liquor-liquor-sales.}
  \end{abstract}
  
 \end{frontmatter}

\section{Problem}\label{problem}

Liquor sales are highly variable and the objective of this report is to
create a statistical model for the volume sold of liquor in gallons by
region within the state of Iowa. This will help us predict inventory and
assist wholesale distributors to plan for predicted volume of
distribution.

\section{Introduction}\label{introduction}

In February, the Distilled Spirits Council (DISCUS), announced that
spirits had an estimated retail sales of nearly \$72 billion in 2015.
Additionally, DISCUS credits the continuous growth of the distilled
spirits industry to several key factors - continuous fascination with
American Whiskeys in the United States and abroad, innovations in
flavors, permiumization across all spirits categories leading to
consumer interest, improved regulatory and tax environment resulting in
expanded market access and a relatively low number of state tax threats,
and the growth of small distillers, which expanded grassroots and
overall interest in the spirits category Del Buono (2016).

This establishes that spirit sales in the Unites States is a valuable
market worth exploring for a more detailed and statistical understanding
of sales and volume. We hope to more throughly understand what impact
regional and season impacts might have on liquor sales. We will limit
the analysis to Iowa which has also reported sales at a record pace
during the last half of 2000 Boshart (2001). While this older
information we do have data up to 2016 to review.

\section{Research Background (Literature
Review)}\label{research-background-literature-review}

Our goal is inventory prediction.

\section{Methodology}\label{methodology}

The data set contains \texttt{5} number of variables,\texttt{County},
\texttt{Category\ name}, \texttt{Bottles\ Sold}, \texttt{Sale\ Dollars},
and the dependent variable is \texttt{Volume\ Sold\ in\ Gallons}. The
initial data set is very large as it includes sales by location and is
very granular. The size of the initial data set has every liquor
transaction from 2012 to present so it approaches \textasciitilde{}2gb.
For the purposes of this report, to analyze a data set this large is not
feasible. Therefore, we reduced the number of varibles and summarized to
a higher level regional aggregate.

We first looked at the top 5 liquor categories for each year by number
of bottles sold. In 2015, the top categories were ``X'' and
interestingly straight burbon appears to have more sales in 2015 than
2014 which coincides with the literature of growing whiskey sales ``ADD
REFERENCE''.

discuss the key aspects of your problem, data set and regression
model(s). Given that you are working on real-world data, explain at a
high-level your exploratory data analysis, how you prepared the data for
regression modeling, your process for building regression models, and
your model selection.

\section{Experimentation and Results}\label{experimentation-and-results}

\section{Data Aquisition}\label{data-aquisition}

The dataset contains the spirits purchase information of Iowa Class
``E'' liquor licensees by product and date of purchase from January 1,
2012 to current. The data set is provided by the Iowa Department of
Commerce, Alcoholic Beverages Division,
\href{https://data.iowa.gov/Economy/Iowa-Liquor-Sales/m3tr-qhgy}{click
here} to view the data set at Data.Iowa.Gov.

describe the specifics of what you did (data exploration, data
preparation, model building, model selection, model evaluation, etc.),
and what you found out (statistical analyses, interpretation and
discussion of the results, etc.).

\section{Discussion and Conclusions}\label{discussion-and-conclusions}

In another study conducted in 2012 in Idaho, the monthly revenue
generated was examined rather than the yearly revenue generated. The
continued growth was rather owed to the number of weekends a month has
(five instead of four) and to the higher prices in neighboring states.
In Washington, the voters approved an initiative that led the state to
sell its liquor stores and add new distributor and retail fees, making
prices in the neighboring states (Idaho and Oregon) look better. There
were no changes made in marketing or pricing in response to the
regulatory shift in Washington ({\textbf{???}}). Further research into
the proximity of our counties to states and towns with higher prices and
regulation may provide more insight into sales and volume of liquor
sold. Additionally, reviewing the data by identifying months that has 5
weekends instead of four could provide further insights.

conclude your findings, limitations, and suggest areas for future work.

\newpage

\section{Appendices}\label{appendices}

\section{Supplemental tables and/or
figures.}\label{supplemental-tables-andor-figures.}

\section{Session Info}\label{session-info}

\begin{itemize}\raggedright
  \item R version 3.3.2 (2016-10-31), \verb|x86_64-w64-mingw32|
  \item Locale: \verb|LC_COLLATE=English_United States.1252|, \verb|LC_CTYPE=English_United States.1252|, \verb|LC_MONETARY=English_United States.1252|, \verb|LC_NUMERIC=C|, \verb|LC_TIME=English_United States.1252|
  \item Base packages: base, datasets, graphics, grDevices,
    methods, stats, utils
  \item Other packages: dplyr~0.5.0, ggplot2~2.2.0, magrittr~1.5,
    pacman~0.4.1
  \item Loaded via a namespace (and not attached): assertthat~0.1,
    backports~1.0.4, colorspace~1.3-1, DBI~0.5-1, digest~0.6.10,
    evaluate~0.10, grid~3.3.2, gtable~0.2.0, htmltools~0.3.5,
    knitr~1.15.1, lazyeval~0.2.0, munsell~0.4.3, plyr~1.8.4,
    R6~2.2.0, Rcpp~0.12.8, rmarkdown~1.2, rprojroot~1.1,
    rticles~0.2, scales~0.4.1, stringi~1.1.2, stringr~1.1.0,
    tibble~1.2, tools~3.3.2, yaml~2.1.14
\end{itemize}

\section{R statistical programming
code.}\label{r-statistical-programming-code.}

Please see
\href{https://github.com/ChristopheHunt/DATA-621-Group-1/blob/master/Final\%20Project/Final\%20Project.Rmd}{Final
Project.rmd} on GitHub for source code.

https://github.com/ChristopheHunt/DATA-621-Group-1/blob/master/Final\%20Project/Final\%20Project.Rmd

\section*{References}\label{references}
\addcontentsline{toc}{section}{References}

\hypertarget{refs}{}
\hypertarget{ref-IowaSetsRecord2}{}
Boshart, Rod. 2001. ``Liquor Sales in Iowa Set Record.'' \emph{Gazette}.

\hypertarget{ref-KeepingSpiritsHigh1}{}
Del Buono, Amanda. 2016. ``Keeping Spirits High.'' \emph{Beverage
Industry} 107.4: 14--16, 18.

\end{document}


