\documentclass[]{elsarticle} %review=doublespace preprint=single 5p=2 column
%%% Begin My package additions %%%%%%%%%%%%%%%%%%%
\usepackage[hyphens]{url}
\usepackage{lineno} % add
\providecommand{\tightlist}{%
  \setlength{\itemsep}{0pt}\setlength{\parskip}{0pt}}

\bibliographystyle{elsarticle-harv}
\biboptions{sort&compress} % For natbib
\usepackage{graphicx}
\usepackage{booktabs} % book-quality tables
%% Redefines the elsarticle footer
%\makeatletter
%\def\ps@pprintTitle{%
% \let\@oddhead\@empty
% \let\@evenhead\@empty
% \def\@oddfoot{\it \hfill\today}%
% \let\@evenfoot\@oddfoot}
%\makeatother

% A modified page layout
\textwidth 6.75in
\oddsidemargin -0.15in
\evensidemargin -0.15in
\textheight 9in
\topmargin -0.5in
%%%%%%%%%%%%%%%% end my additions to header

\usepackage[T1]{fontenc}
\usepackage{lmodern}
\usepackage{amssymb,amsmath}
\usepackage{ifxetex,ifluatex}
\usepackage{fixltx2e} % provides \textsubscript
% use upquote if available, for straight quotes in verbatim environments
\IfFileExists{upquote.sty}{\usepackage{upquote}}{}
\ifnum 0\ifxetex 1\fi\ifluatex 1\fi=0 % if pdftex
  \usepackage[utf8]{inputenc}
\else % if luatex or xelatex
  \usepackage{fontspec}
  \ifxetex
    \usepackage{xltxtra,xunicode}
  \fi
  \defaultfontfeatures{Mapping=tex-text,Scale=MatchLowercase}
  \newcommand{\euro}{€}
\fi
% use microtype if available
\IfFileExists{microtype.sty}{\usepackage{microtype}}{}
\usepackage{color}
\usepackage{fancyvrb}
\newcommand{\VerbBar}{|}
\newcommand{\VERB}{\Verb[commandchars=\\\{\}]}
\DefineVerbatimEnvironment{Highlighting}{Verbatim}{commandchars=\\\{\}}
% Add ',fontsize=\small' for more characters per line
\usepackage{framed}
\definecolor{shadecolor}{RGB}{248,248,248}
\newenvironment{Shaded}{\begin{snugshade}}{\end{snugshade}}
\newcommand{\KeywordTok}[1]{\textcolor[rgb]{0.13,0.29,0.53}{\textbf{{#1}}}}
\newcommand{\DataTypeTok}[1]{\textcolor[rgb]{0.13,0.29,0.53}{{#1}}}
\newcommand{\DecValTok}[1]{\textcolor[rgb]{0.00,0.00,0.81}{{#1}}}
\newcommand{\BaseNTok}[1]{\textcolor[rgb]{0.00,0.00,0.81}{{#1}}}
\newcommand{\FloatTok}[1]{\textcolor[rgb]{0.00,0.00,0.81}{{#1}}}
\newcommand{\ConstantTok}[1]{\textcolor[rgb]{0.00,0.00,0.00}{{#1}}}
\newcommand{\CharTok}[1]{\textcolor[rgb]{0.31,0.60,0.02}{{#1}}}
\newcommand{\SpecialCharTok}[1]{\textcolor[rgb]{0.00,0.00,0.00}{{#1}}}
\newcommand{\StringTok}[1]{\textcolor[rgb]{0.31,0.60,0.02}{{#1}}}
\newcommand{\VerbatimStringTok}[1]{\textcolor[rgb]{0.31,0.60,0.02}{{#1}}}
\newcommand{\SpecialStringTok}[1]{\textcolor[rgb]{0.31,0.60,0.02}{{#1}}}
\newcommand{\ImportTok}[1]{{#1}}
\newcommand{\CommentTok}[1]{\textcolor[rgb]{0.56,0.35,0.01}{\textit{{#1}}}}
\newcommand{\DocumentationTok}[1]{\textcolor[rgb]{0.56,0.35,0.01}{\textbf{\textit{{#1}}}}}
\newcommand{\AnnotationTok}[1]{\textcolor[rgb]{0.56,0.35,0.01}{\textbf{\textit{{#1}}}}}
\newcommand{\CommentVarTok}[1]{\textcolor[rgb]{0.56,0.35,0.01}{\textbf{\textit{{#1}}}}}
\newcommand{\OtherTok}[1]{\textcolor[rgb]{0.56,0.35,0.01}{{#1}}}
\newcommand{\FunctionTok}[1]{\textcolor[rgb]{0.00,0.00,0.00}{{#1}}}
\newcommand{\VariableTok}[1]{\textcolor[rgb]{0.00,0.00,0.00}{{#1}}}
\newcommand{\ControlFlowTok}[1]{\textcolor[rgb]{0.13,0.29,0.53}{\textbf{{#1}}}}
\newcommand{\OperatorTok}[1]{\textcolor[rgb]{0.81,0.36,0.00}{\textbf{{#1}}}}
\newcommand{\BuiltInTok}[1]{{#1}}
\newcommand{\ExtensionTok}[1]{{#1}}
\newcommand{\PreprocessorTok}[1]{\textcolor[rgb]{0.56,0.35,0.01}{\textit{{#1}}}}
\newcommand{\AttributeTok}[1]{\textcolor[rgb]{0.77,0.63,0.00}{{#1}}}
\newcommand{\RegionMarkerTok}[1]{{#1}}
\newcommand{\InformationTok}[1]{\textcolor[rgb]{0.56,0.35,0.01}{\textbf{\textit{{#1}}}}}
\newcommand{\WarningTok}[1]{\textcolor[rgb]{0.56,0.35,0.01}{\textbf{\textit{{#1}}}}}
\newcommand{\AlertTok}[1]{\textcolor[rgb]{0.94,0.16,0.16}{{#1}}}
\newcommand{\ErrorTok}[1]{\textcolor[rgb]{0.64,0.00,0.00}{\textbf{{#1}}}}
\newcommand{\NormalTok}[1]{{#1}}
\usepackage{longtable}
\usepackage{graphicx}
% We will generate all images so they have a width \maxwidth. This means
% that they will get their normal width if they fit onto the page, but
% are scaled down if they would overflow the margins.
\makeatletter
\def\maxwidth{\ifdim\Gin@nat@width>\linewidth\linewidth
\else\Gin@nat@width\fi}
\makeatother
\let\Oldincludegraphics\includegraphics
\renewcommand{\includegraphics}[1]{\Oldincludegraphics[width=\maxwidth]{#1}}
\ifxetex
  \usepackage[setpagesize=false, % page size defined by xetex
              unicode=false, % unicode breaks when used with xetex
              xetex]{hyperref}
\else
  \usepackage[unicode=true]{hyperref}
\fi
\hypersetup{breaklinks=true,
            bookmarks=true,
            pdfauthor={},
            pdftitle={Regional Liquor Sales in Des Moines, Iowa},
            colorlinks=true,
            urlcolor=blue,
            linkcolor=magenta,
            pdfborder={0 0 0}}
\urlstyle{same}  % don't use monospace font for urls
\setlength{\parindent}{0pt}
\setlength{\parskip}{6pt plus 2pt minus 1pt}
\setlength{\emergencystretch}{3em}  % prevent overfull lines
\setcounter{secnumdepth}{0}
% Pandoc toggle for numbering sections (defaults to be off)
\setcounter{secnumdepth}{0}
% Pandoc header


\usepackage[nomarkers]{endfloat}

\begin{document}
\begin{frontmatter}

  \title{Regional Liquor Sales in Des Moines, Iowa}
    \author[CUNY School of Professional Studies]{Christophe Hunt\corref{c1}}
   \ead{christophe.hunt@spsmail.cuny.edu} 
   \cortext[c1]{Authors}
    \author[CUNY School of Professional Studies]{Senthil Dhanapal}
   \ead{senthil.dhanapal@spsmail.cuny.edu} 
  
    \author[CUNY School of Professional Studies]{Yadu Chittampalli}
   \ead{yadu.chittampalli@spsmail.cuny.edu} 
  
      \address[CUNY School of Professional Studies]{CUNY School of Professional Studies, Data Analytics, New York, NY}
  
  \begin{abstract}
  This is the abstract.
  
  It consists of two paragraphs
  
  \section{Keywords: Liquor Sales, Naive
  Forecast*}\label{keywords-liquor-sales-naive-forecast}
  \end{abstract}
  
 \end{frontmatter}

\section{Problem}\label{problem}

The objective of this report is to create a statistical model for the
volume sold of liquor in gallons and the profit dollars in the City of
Des Moines which is within the state of Iowa. This can help us make
informed decisions on inventory production, sales, and assist wholesale
distributors to plan for the predicted volume of distribution.

\section{Introduction}\label{introduction}

In February, the Distilled Spirits Council (DISCUS), announced that
spirits had an estimated retail sales of nearly \$72 billion in 2015.
Additionally, DISCUS credits the continuous growth of the distilled
spirits industry to several key factors - continuous fascination with
American Whiskeys in the United States and abroad, innovations in
flavors, permutation across all spirits categories leading to consumer
interest, improved regulatory and tax environment resulting in expanded
market access and a relatively low number of state tax threats, and the
growth of small distillers, which expanded grassroots and overall
interest in the spirits category Del Buono (2016).

This establishes that spirit sales in the Unites States is a valuable
market worth exploring for a more detailed and statistical understanding
of sales and volume. We hope to more thoroughly understand what impact
specific store sights may have accounting for the seasonal impact in
November that might effect liquor sales. We will limit the analysis to
the City of Des Moines for only whisky sales in the month of November.
In 2000 the State of Iowa reported sales at a record pace during the
last half of 2000 Boshart (2001). The later part of the year has an
increase in sales so planning to meet capacity is a suitable goal for
any company. Our years of interest for this analysis will be the month
of November for 2015 and 2016.

\section{Research Background}\label{research-background}

The main goal that has to be achieved in inventory prediction is
increasing the efficiency without decreasing the service value offered
to the customers. When managing the levels of inventory, it is important
to maintain moderate level(s) - not too high and not too low. If the
inventory level is excessive, business funds can get wasted. These funds
would not be able to be used for any other purpose, thus involving an
opportunity cost. The costs of shortage, handling insurance, recording
and inspection would proportionately increase along with inventory
volume, thus impairing profitability.

On the other hand, low level(s) of inventory may result in frequent
interruptions in the production schedule resulting in under-utilization
of capacity and lower sales. When making predictions about orders that
should be placed, assumptions are made as follows - uncertainty always
exists regardless of the method(s) used, new technologies cannot always
be forecasted for which paradigms do not exist, and social policy will
be formulated where the future would be affected, changing the accuracy
of the forecast.

One useful method for predicting inventories is the extrapolation of
trends. In this method, trends and cycles in the historical data are
examined and mathematical techniques are used to extrapolate to the
future. The model chosen for forecasting would depend on the historical
data. One of the most common models used in this method is
decomposition, where historical data is separated into trend, seasonal,
and random components. As a result, forecasts are produced using
``turning point analysis''. Other examples of models used are adaptive
filtering, Box-Jenkins analysis, simple linear regression, curve
fitting, and weighted smoothing.

According to Makridakis, ``Judgmental forecasting is superior to
mathematical models, although there are several forecasting applications
where computer-generated models would be more feasible.'' When inventory
levels for bulk-quantity items would need to be forecasted monthly by
large manufacturing companies, generating models through computer
software would be more efficient.

http://www.statpac.org/research-library/forecasting.htm

Forecasting the demand of a product is very essential in predicting the
order quantity. As a result, a data bank is created, helping the
decision makers settle targets, create plans, and demonstrate changes in
the business setting. Two different methods are utilized in the
investigation of the future demand - quantitative and qualitative. In
quantitative methods, mathematical consistencies in the history are
searched for. Two subcategories exist in quantitative methods - time
series models and correlation models. On the other hand, qualitative
methods are based on the opinions that people have had about the product
in the past based on their experiences, premonitions, and emotions.

However, when the most suitable forecast model gets selected, it is not
necessarily based solely on quantitative or qualitative variables. The
forecast model can even combine several models.

http://www.slideshare.net/AbhishekKumar159/demand-forecast-process-and-inventory-management

There was a study conducted by the Beer Industry Electronic Commerce
Coalition. The purpose of this study was to highlight, explain, and
create future considerations for the advancement of forecasting future
beer sales and the process around how distributors and suppliers can
collaborate to achieve this goal. Weekly forecasts can be created by
supplier, brand, package, etc. for up to fifteen weeks into the future
if necessary. It is important to be aware of the product trends so that
specific situations can be explained to various employees of the company
at any time.

https://www.nbwa.org/sites/default/files/hitting\_the\_mark.pdf

\section{Methodology}\label{methodology}

Our initial data set is sufficiently large in that it includes sales by
individual stores and the invoices for each store. The reason for the
large size of the initial data set is due to it including every liquor
transaction from 2012 to present in Iowa, so it approaches 2.68 GB. For
the purposes of this analysis, to analyze a data set this large is not
feasible. Therefore, we reduced the number of variables and summarized
to a regional aggregate.

Additionally, we looked into the top 10 liquor categories for each year
by number of bottles sold. In 2015, the top categories were American
Cocktails, Blended Whiskies, Canadian Whiskies, Imported Vodka, Puerto
Rico \& Virgin Islands Rum, Spiced Rum, Straight Bourbon Whiskies,
Tequila, Vodka 80 Proof, and Whiskey Liqueur. Interestingly straight
bourbon appears to have more sales in 2015 than 2014 which coincides
with the literature of strong growing whiskey sales for every whiskey
segment (Anonymous (2016)). We decided to focus on whiskey do to its
strong sales and growing interest in the US.

Furthermore, we looked into the sales by \texttt{County} for Iowa

We initially attempted to model for a dependent variable of
\texttt{Volume\ Sold\ in\ Gallons} by the top Counties and top liquor
categories, however, the distribution of this variable becomes
over-dispersed and negatively skewed when aggregating the data set and
therefore we were unsuccessful in modeling this variable across regions.
Our hope was that if we could model for the gallons sold by entirety of
Iowa we could more accurately predict our planned inventory and
anticipate production goals. We were more successful in limiting our
regional analysis to one city and the stores within that city.

Therefore, our final evaluation data set is the following subset of
variables for largest city in Iowa of Des Moines as follows;
\texttt{Vendor\ Name}, \texttt{Pack} (pack size of bottles sold)
\texttt{Bottle\ Volume}, \texttt{State\ Bottle\ Cost},
\texttt{State\ Bottle\ Retail}, \texttt{Sales\ Dollars} (Total sales),
and our dependent variables are \texttt{Volume\ sold\ in\ Gallons} and
\texttt{Profit\ Dollars}.

We then sought to model the mean price per bottle, by modeling the price
per bottle we can make pricing decisions based on region and the volume
we plan to sell.

discuss the key aspects of your problem, data set and regression
model(s). Given that you are working on real-world data, explain at a
high-level your exploratory data analysis, how you prepared the data for
regression modeling, your process for building regression models, and
your model selection.

\section{Experimentation and Results}\label{experimentation-and-results}

\section{Model 1}\label{model-1}

\section{Forward step}\label{forward-step}

\begin{table}[!htbp] \centering 
  \caption{Forward Selection Linear Model 1} 
  \label{} 
\normalsize 
\begin{tabular}{@{\extracolsep{5pt}}lc} 
\\[-1.8ex]\hline 
\hline \\[-1.8ex] 
 & \multicolumn{1}{c}{\textit{Dependent variable:}} \\ 
\cline{2-2} 
\\[-1.8ex] & Bottles.Sold \\ 
\hline \\[-1.8ex] 
 Constant & $-$31.762$^{***}$ (7.570) \\ 
  Volume.Sold..Gallons. & 4.093$^{***}$ (0.162) \\ 
  Category.NameCANADIAN WHISKIES & 36.254$^{***}$ (6.601) \\ 
  Category.NameIRISH WHISKIES & $-$4.874 (8.077) \\ 
  Category.NameSCOTCH WHISKIES & $-$8.116 (8.189) \\ 
  Category.NameSINGLE BARREL BOURBON WHISKIES & 12.384 (14.612) \\ 
  Category.NameSTRAIGHT BOURBON WHISKIES & $-$3.416 (6.985) \\ 
  Category.NameSTRAIGHT RYE WHISKIES & 17.005$^{*}$ (9.423) \\ 
  Category.NameTENNESSEE WHISKIES & $-$1.311 (7.125) \\ 
  State.Bottle.Retail & 0.079$^{***}$ (0.011) \\ 
  Pack & 2.760$^{***}$ (0.367) \\ 
  Sale..Dollars. & $-$0.004$^{**}$ (0.002) \\ 
 \hline \\[-1.8ex] 
Observations & 376 \\ 
R$^{2}$ & 0.984 \\ 
Adjusted R$^{2}$ & 0.984 \\ 
Residual Std. Error & 37.162 (df = 364) \\ 
F Statistic & 2,076.152$^{***}$ (df = 11; 364)  (p = 0.000) \\ 
\hline 
\hline \\[-1.8ex] 
\textit{Note:}  & \multicolumn{1}{r}{$^{*}$p$<$0.1; $^{**}$p$<$0.05; $^{***}$p$<$0.01} \\ 
\end{tabular} 
\end{table}

\section{Volume.Sold..Gallons=One unit of increase in volume sold will
increase the bottles sold by 4.10
units}\label{volume.sold..gallonsone-unit-of-increase-in-volume-sold-will-increase-the-bottles-sold-by-4.10-units}

\section{Category.Name}\label{category.name}

\section{CANADIAN WHISKIES is the only one that is
significant.}\label{canadian-whiskies-is-the-only-one-that-is-significant.}

\section{CANDIAN WHISKIES vs BLENDED WHISKIES coefficient is
5.492}\label{candian-whiskies-vs-blended-whiskies-coefficient-is-5.492}

\section{State.Bottle.Retail=One unit of increase in State.Bottle.Retail
will increase the bottles sold by 7.133
units}\label{state.bottle.retailone-unit-of-increase-in-state.bottle.retail-will-increase-the-bottles-sold-by-7.133-units}

\section{Pack=One unit of increase in Pack will increase the bottles
sold by 7.133
units}\label{packone-unit-of-increase-in-pack-will-increase-the-bottles-sold-by-7.133-units}

\section{Sale..Dollars.=It is slightly significant. One unit of increase
in Sale..Dollars. will decrease the bottles sold by 2.189
units}\label{sale..dollars.it-is-slightly-significant.-one-unit-of-increase-in-sale..dollars.-will-decrease-the-bottles-sold-by-2.189-units}

\section{\texorpdfstring{Formula = -4.196 +
25.314\emph{Volume.Sold..Gallons. + 5.492 } (Category.Name=``CANADIAN
WHISKIES'' )
+}{Formula = -4.196 + 25.314Volume.Sold..Gallons. + 5.492  (Category.Name=CANADIAN WHISKIES ) +}}\label{formula--4.196-25.314volume.sold..gallons.-5.492-category.namecanadian-whiskies}

\section{7.133 * (State.Bottle.Retail) + 7.521 * (Pack) - 2.189 *
Sale..Dollars.}\label{state.bottle.retail-7.521-pack---2.189-sale..dollars.}

\newpage

\section{Data Acquisition}\label{data-acquisition}

The data set contains the spirits purchase information of Iowa Class
``E'' liquor licensees by product and date of purchase from January 1,
2012 to current. The data set is provided by the Iowa Department of
Commerce, Alcoholic Beverages Division,
\href{https://data.iowa.gov/Economy/Iowa-Liquor-Sales/m3tr-qhgy}{click
here} to view the data set at Data.Iowa.Gov.

As previously discussed, the data set is 2.68 GB in total size and much
to large to use in a meaningful model.

We reviewed the liquor sales by gallons sold per year by Liquor
Category. Initially, we viewed the top 5 Liquor Categories by volume
sold but there were large disparaties between years, suggesting that the
top 5 change often and is likely due to changing consumer tastes. We do
see a more stable set of liquor categories for the top 10 category which
suggests that while tastes may change we don't see large movements in
liquor categories at this level. We focused our attention on the whiskey
categories.

\includegraphics{Final_Project_files/figure-latex/unnamed-chunk-6-1.pdf}

Our first attempt was to use a Poisson regression due to the
over-dispersion created by aggregating the data set. However, the
distribution was far to negatively skewed to fit a poission
distribution. We therefore chose

\subsection{\texorpdfstring{\protect\includegraphics{Final_Project_files/figure-latex/unnamed-chunk-8-1.pdf}}{}}\label{section}

\begin{longtable}[]{@{}cccccccc@{}}
\toprule
min & max & median & mean & sd & skewness & kurtosis &
method\tabularnewline
\midrule
\endhead
2814 & 962072 & 24111.5 & 53022.46 & 92140.34 & 5.928606 & 52.56773 &
unbiased\tabularnewline
\bottomrule
\end{longtable}

\includegraphics{Final_Project_files/figure-latex/unnamed-chunk-9-1.pdf}

describe the specifics of what you did (data exploration, data
preparation, model building, model selection, model evaluation, etc.),
and what you found out (statistical analyses, interpretation and
discussion of the results, etc.).

\section{Discussion and Conclusions}\label{discussion-and-conclusions}

In two studies that were conducted, the statistical data from the
previous time interval was used to predict what would happen in the next
time interval. One study was for pharmaceutical distribution companies.
The other study involved data mining models.

In one study involving pharmaceutical distribution companies, the
purpose was to propose a novel method to forecast the sales of the
companies. Network-based analysis was conducted to find clique sets and
group members and to use the sales data of comembers. The reason for
this was the lack of sufficient historic sales records of each drug.

Three methods were used to build time series models forecasting sales -
ARIMA methodology, neural network, and an advanced hybrid neural network
approach. The performance of the proposed method was evaluated using a
real dataset provided by one of the leading pharmacy distribution
companies in Iran. The results of the evaluation indicated that the
proposed method can cope with the low number of past records while
accurately forecasting medicine sales.

After exploratory analysis was done on the data, it was concluded that
most medicines had different and specific characteristics and sales
behavior, it was impractical to make a single prediction model for all
medicines, and most sales records had nonlinear relationships.

The reason why the hybrid neural network method was carried out was due
to the fact that it is not acceptable to apply a fully linear or
nonlinear model on sales data.

The two forecast error measures that were used to evaluate and compare
model performance were mean squared error and mean absoute error. The
performance of the predicted data was significantly improved when the
past records of comembers were used.

http://onesearch.cuny.edu/primo\_library/libweb/action/display.do?tabs=detailsTab\&ct=display\&fn=search\&doc=TN\_gale\_ofa425455991\&indx=3\&recIds=TN\_gale\_ofa425455991\&recIdxs=2\&elementId=2\&renderMode=poppedOut\&displayMode=full\&frbrVersion=2\&frbg=\&\&dscnt=0\&scp.scps=scope\%3A\%28CUNY\_BEPRESS\%29\%2Cscope\%3A\%28BB\%29\%2Cscope\%3A\%28BB\_LIBGUIDES\%29\%2Cscope\%3A\%28AL\%29\%2Cprimo\_central\_multiple\_fe\&tb=t\&mode=Basic\&vid=bb\&srt=rank\&tab=default\_tab\&dum=true\&vl(freeText0)=inventory\%20prediction\%20sales\&dstmp=1481346949259

The other study that examined prediction-based inventory optimization
using data mining models, the Back propagation neural network was used
for training the prediction model. The idea that gave rise to this
method of inventory prediction was the idea that the demand of marketing
is viewed as the foundation of inventory management.

On the basis of the prediction result, a simple and concise inventory
policy was established. Following this, the historic sales data was used
to estimate a normal distribution of demand and to calculate the
inventory cost with inventory strategy.

Two models (Back Propagation Neural Network and Support Vector
Regression) were established using three input variables (historical
sales data, the frequency of searching the commodity, and the click
volume of the commodity page).

When the back-propagation neural network method was used, there was more
accuracy shown in the performance because the predicted values almost
matched the actual values in the graphs.

The mean absolute percentage error calculated for this model is 0.06.

http://onesearch.cuny.edu/primo\_library/libweb/action/display.do?frbrVersion=2\&tabs=detailsTab\&ct=display\&fn=search\&doc=TN\_ieee10.1109\%2fCSO.2014.118\&indx=10\&recIds=TN\_ieee10.1109\%2fCSO.2014.118\&recIdxs=9\&elementId=9\&renderMode=poppedOut\&displayMode=full\&frbrVersion=2\&frbg=\&\&dscnt=0\&scp.scps=scope\%3A\%28CUNY\_BEPRESS\%29\%2Cscope\%3A\%28BB\%29\%2Cscope\%3A\%28BB\_LIBGUIDES\%29\%2Cscope\%3A\%28AL\%29\%2Cprimo\_central\_multiple\_fe\&tb=t\&mode=Basic\&vid=bb\&srt=rank\&tab=default\_tab\&dum=true\&vl(freeText0)=inventory\%20prediction\%20sales\&dstmp=1481412549638

In another study conducted in 2012 in Idaho, the monthly revenue
generated was examined rather than the yearly revenue generated. The
continued growth was rather owed to the number of weekends a month has
(five instead of four) and to the higher prices in neighboring states.
In Washington, the voters approved an initiative that led the state to
sell its liquor stores and add new distributor and retail fees, making
prices in the neighboring states (Idaho and Oregon) look better. There
were no changes made in marketing or pricing in response to the
regulatory shift in Washington ({\textbf{???}}). Further research into
the proximity of our counties to states and towns with higher prices and
regulation may provide more insight into sales and volume of liquor
sold. Additionally, reviewing the data by identifying months that has 5
weekends instead of four could provide further insights.

conclude your findings, limitations, and suggest areas for future work.

\newpage

\section{Appendices}\label{appendices}

\section{Supplemental tables and/or
figures.}\label{supplemental-tables-andor-figures.}

\begin{spacing}{0.7}
\begin{center}\textbf{ dfLiquorSales \\ 14 Variables~~~~~ 376 ~Observations}\end{center}
\smallskip\hrule\smallskip{\small
\noindent\textbf{X}\setlength{\unitlength}{0.001in}\hfill\begin{picture}(1.5,.1)(1500,0)\linethickness{0.6pt}
\put(0,0){\line(0,1){40}}
\put(20,0){\line(0,1){100}}
\put(39,0){\line(0,1){100}}
\put(59,0){\line(0,1){100}}
\put(79,0){\line(0,1){100}}
\put(99,0){\line(0,1){100}}
\put(118,0){\line(0,1){100}}
\put(138,0){\line(0,1){100}}
\put(158,0){\line(0,1){100}}
\put(178,0){\line(0,1){100}}
\put(197,0){\line(0,1){100}}
\put(217,0){\line(0,1){100}}
\put(237,0){\line(0,1){100}}
\put(257,0){\line(0,1){100}}
\put(276,0){\line(0,1){100}}
\put(296,0){\line(0,1){100}}
\put(316,0){\line(0,1){100}}
\put(336,0){\line(0,1){100}}
\put(355,0){\line(0,1){100}}
\put(375,0){\line(0,1){100}}
\put(395,0){\line(0,1){100}}
\put(414,0){\line(0,1){100}}
\put(434,0){\line(0,1){100}}
\put(454,0){\line(0,1){100}}
\put(474,0){\line(0,1){100}}
\put(493,0){\line(0,1){100}}
\put(513,0){\line(0,1){100}}
\put(533,0){\line(0,1){100}}
\put(553,0){\line(0,1){100}}
\put(572,0){\line(0,1){100}}
\put(592,0){\line(0,1){100}}
\put(612,0){\line(0,1){100}}
\put(632,0){\line(0,1){100}}
\put(651,0){\line(0,1){100}}
\put(671,0){\line(0,1){100}}
\put(691,0){\line(0,1){100}}
\put(711,0){\line(0,1){100}}
\put(730,0){\line(0,1){100}}
\put(750,0){\line(0,1){100}}
\put(770,0){\line(0,1){100}}
\put(789,0){\line(0,1){100}}
\put(809,0){\line(0,1){100}}
\put(829,0){\line(0,1){100}}
\put(849,0){\line(0,1){100}}
\put(868,0){\line(0,1){100}}
\put(888,0){\line(0,1){100}}
\put(908,0){\line(0,1){100}}
\put(928,0){\line(0,1){100}}
\put(947,0){\line(0,1){100}}
\put(967,0){\line(0,1){100}}
\put(987,0){\line(0,1){100}}
\put(1007,0){\line(0,1){100}}
\put(1026,0){\line(0,1){100}}
\put(1046,0){\line(0,1){100}}
\put(1066,0){\line(0,1){100}}
\put(1086,0){\line(0,1){100}}
\put(1105,0){\line(0,1){100}}
\put(1125,0){\line(0,1){100}}
\put(1145,0){\line(0,1){100}}
\put(1164,0){\line(0,1){100}}
\put(1184,0){\line(0,1){100}}
\put(1204,0){\line(0,1){100}}
\put(1224,0){\line(0,1){100}}
\put(1243,0){\line(0,1){100}}
\put(1263,0){\line(0,1){100}}
\put(1283,0){\line(0,1){100}}
\put(1303,0){\line(0,1){100}}
\put(1322,0){\line(0,1){100}}
\put(1342,0){\line(0,1){100}}
\put(1362,0){\line(0,1){100}}
\put(1382,0){\line(0,1){100}}
\put(1401,0){\line(0,1){100}}
\put(1421,0){\line(0,1){100}}
\put(1441,0){\line(0,1){100}}
\put(1461,0){\line(0,1){100}}
\put(1480,0){\line(0,1){80}}
\end{picture}

{\smaller[2]
\begin{tabular}{ rrrrrrrrrrrrr }
n&missing&distinct&Info&Mean&Gmd&.05&.10&.25&.50&.75&.90&.95 \\
376&0&376&1&188.5&125.7& 19.75& 38.50& 94.75&188.50&282.25&338.50&357.25 \end{tabular}
\begin{verbatim}

lowest :   1   2   3   4   5, highest: 372 373 374 375 376
\end{verbatim}
}
\smallskip\hrule\smallskip
\noindent\textbf{Month}

{\smaller
\begin{tabular}{ rrrrrr }
n&missing&distinct&Info&Mean&Gmd \\
376&0&1&0&11&0 \end{tabular}
\begin{verbatim}
              
Value       11
Frequency  376
Proportion   1
\end{verbatim}
}
\smallskip\hrule\smallskip
\noindent\textbf{Year}

{\smaller
\begin{tabular}{ rrrrrr }
n&missing&distinct&Info&Mean&Gmd \\
376&0&1&0&2015&0 \end{tabular}
\begin{verbatim}
               
Value      2015
Frequency   376
Proportion    1
\end{verbatim}
}
\smallskip\hrule\smallskip
\noindent\textbf{City}

{\smaller
\begin{tabular}{ rrrr }
n&missing&distinct&value \\
376&0&1&DES MOINES \end{tabular}
\begin{verbatim}
                     
Value      DES MOINES
Frequency         376
Proportion          1
\end{verbatim}
}
\smallskip\hrule\smallskip
\noindent\textbf{Category.Name}\setlength{\unitlength}{0.001in}\hfill\begin{picture}(1.5,.1)(1500,0)\linethickness{0.6pt}
\put(0,0){\line(0,1){86}}
\put(188,0){\line(0,1){100}}
\put(375,0){\line(0,1){55}}
\put(562,0){\line(0,1){58}}
\put(750,0){\line(0,1){11}}
\put(938,0){\line(0,1){93}}
\put(1125,0){\line(0,1){38}}
\put(1312,0){\line(0,1){89}}
\end{picture}

{\smaller
\begin{tabular}{ rrr }
n&missing&distinct \\
376&0&8 \end{tabular}
\begin{verbatim}

BLENDED WHISKIES (61, 0.162), CANADIAN WHISKIES (71, 0.189), IRISH WHISKIES (39, 0.104),
SCOTCH WHISKIES (41, 0.109), SINGLE BARREL BOURBON WHISKIES (8, 0.021), STRAIGHT BOURBON
WHISKIES (66, 0.176), STRAIGHT RYE WHISKIES (27, 0.072), TENNESSEE WHISKIES (63, 0.168)
\end{verbatim}
}
\smallskip\hrule\smallskip
\noindent\textbf{Store.Number}\setlength{\unitlength}{0.001in}\hfill\begin{picture}(1.5,.1)(1500,0)\linethickness{0.6pt}
\put(0,0){\line(0,1){100}}
\put(29,0){\line(0,1){100}}
\put(167,0){\line(0,1){100}}
\put(168,0){\line(0,1){88}}
\put(170,0){\line(0,1){88}}
\put(184,0){\line(0,1){100}}
\put(217,0){\line(0,1){100}}
\put(217,0){\line(0,1){75}}
\put(220,0){\line(0,1){100}}
\put(221,0){\line(0,1){88}}
\put(328,0){\line(0,1){25}}
\put(329,0){\line(0,1){62}}
\put(601,0){\line(0,1){75}}
\put(739,0){\line(0,1){88}}
\put(748,0){\line(0,1){75}}
\put(786,0){\line(0,1){50}}
\put(789,0){\line(0,1){62}}
\put(792,0){\line(0,1){75}}
\put(812,0){\line(0,1){75}}
\put(838,0){\line(0,1){62}}
\put(927,0){\line(0,1){88}}
\put(930,0){\line(0,1){75}}
\put(968,0){\line(0,1){75}}
\put(969,0){\line(0,1){75}}
\put(983,0){\line(0,1){38}}
\put(991,0){\line(0,1){75}}
\put(1034,0){\line(0,1){75}}
\put(1045,0){\line(0,1){62}}
\put(1048,0){\line(0,1){38}}
\put(1069,0){\line(0,1){50}}
\put(1078,0){\line(0,1){62}}
\put(1086,0){\line(0,1){50}}
\put(1136,0){\line(0,1){75}}
\put(1153,0){\line(0,1){75}}
\put(1194,0){\line(0,1){50}}
\put(1194,0){\line(0,1){50}}
\put(1195,0){\line(0,1){50}}
\put(1195,0){\line(0,1){50}}
\put(1196,0){\line(0,1){25}}
\put(1196,0){\line(0,1){50}}
\put(1197,0){\line(0,1){50}}
\put(1197,0){\line(0,1){38}}
\put(1198,0){\line(0,1){38}}
\put(1205,0){\line(0,1){75}}
\put(1208,0){\line(0,1){38}}
\put(1217,0){\line(0,1){50}}
\put(1231,0){\line(0,1){50}}
\put(1294,0){\line(0,1){62}}
\put(1294,0){\line(0,1){62}}
\put(1296,0){\line(0,1){38}}
\put(1296,0){\line(0,1){38}}
\put(1297,0){\line(0,1){62}}
\put(1297,0){\line(0,1){88}}
\put(1298,0){\line(0,1){50}}
\put(1298,0){\line(0,1){50}}
\put(1311,0){\line(0,1){100}}
\put(1321,0){\line(0,1){25}}
\put(1359,0){\line(0,1){25}}
\put(1367,0){\line(0,1){38}}
\put(1383,0){\line(0,1){38}}
\put(1394,0){\line(0,1){88}}
\put(1421,0){\line(0,1){50}}
\put(1424,0){\line(0,1){62}}
\put(1458,0){\line(0,1){75}}
\put(1458,0){\line(0,1){88}}
\put(1459,0){\line(0,1){88}}
\put(1459,0){\line(0,1){88}}
\put(1460,0){\line(0,1){88}}
\put(1461,0){\line(0,1){62}}
\put(1461,0){\line(0,1){62}}
\put(1464,0){\line(0,1){38}}
\put(1468,0){\line(0,1){75}}
\put(1479,0){\line(0,1){50}}
\end{picture}

{\smaller
\begin{tabular}{ rrr }
n&missing&distinct \\
376&0&73 \end{tabular}
\begin{verbatim}

lowest : 2190 2248 2527 2528 2532, highest: 5131 5132 5137 5145 5169
\end{verbatim}
}
\smallskip\hrule\smallskip
\noindent\textbf{Store.Name}\setlength{\unitlength}{0.001in}\hfill\begin{picture}(1.5,.1)(1500,0)\linethickness{0.6pt}
\put(0,0){\line(0,1){44}}
\put(21,0){\line(0,1){44}}
\put(42,0){\line(0,1){22}}
\put(62,0){\line(0,1){67}}
\put(83,0){\line(0,1){78}}
\put(104,0){\line(0,1){89}}
\put(125,0){\line(0,1){89}}
\put(146,0){\line(0,1){22}}
\put(167,0){\line(0,1){56}}
\put(188,0){\line(0,1){78}}
\put(208,0){\line(0,1){33}}
\put(229,0){\line(0,1){56}}
\put(250,0){\line(0,1){67}}
\put(271,0){\line(0,1){67}}
\put(292,0){\line(0,1){22}}
\put(312,0){\line(0,1){67}}
\put(333,0){\line(0,1){67}}
\put(354,0){\line(0,1){89}}
\put(375,0){\line(0,1){89}}
\put(396,0){\line(0,1){78}}
\put(417,0){\line(0,1){78}}
\put(438,0){\line(0,1){89}}
\put(458,0){\line(0,1){89}}
\put(479,0){\line(0,1){67}}
\put(500,0){\line(0,1){78}}
\put(521,0){\line(0,1){89}}
\put(542,0){\line(0,1){44}}
\put(562,0){\line(0,1){33}}
\put(583,0){\line(0,1){33}}
\put(604,0){\line(0,1){56}}
\put(625,0){\line(0,1){44}}
\put(646,0){\line(0,1){67}}
\put(667,0){\line(0,1){67}}
\put(688,0){\line(0,1){56}}
\put(708,0){\line(0,1){100}}
\put(729,0){\line(0,1){56}}
\put(750,0){\line(0,1){78}}
\put(771,0){\line(0,1){78}}
\put(792,0){\line(0,1){78}}
\put(812,0){\line(0,1){44}}
\put(833,0){\line(0,1){22}}
\put(854,0){\line(0,1){44}}
\put(875,0){\line(0,1){44}}
\put(896,0){\line(0,1){56}}
\put(917,0){\line(0,1){33}}
\put(938,0){\line(0,1){44}}
\put(958,0){\line(0,1){44}}
\put(979,0){\line(0,1){44}}
\put(1000,0){\line(0,1){33}}
\put(1021,0){\line(0,1){33}}
\put(1042,0){\line(0,1){44}}
\put(1062,0){\line(0,1){67}}
\put(1083,0){\line(0,1){78}}
\put(1104,0){\line(0,1){67}}
\put(1125,0){\line(0,1){67}}
\put(1146,0){\line(0,1){67}}
\put(1167,0){\line(0,1){67}}
\put(1188,0){\line(0,1){33}}
\put(1208,0){\line(0,1){33}}
\put(1229,0){\line(0,1){56}}
\put(1250,0){\line(0,1){78}}
\put(1271,0){\line(0,1){67}}
\put(1292,0){\line(0,1){44}}
\put(1312,0){\line(0,1){67}}
\put(1333,0){\line(0,1){56}}
\put(1354,0){\line(0,1){56}}
\put(1375,0){\line(0,1){33}}
\put(1396,0){\line(0,1){56}}
\put(1417,0){\line(0,1){78}}
\put(1438,0){\line(0,1){44}}
\put(1458,0){\line(0,1){44}}
\put(1479,0){\line(0,1){33}}
\end{picture}

{\smaller
\begin{tabular}{ rrr }
n&missing&distinct \\
376&0&72 \end{tabular}
\begin{verbatim}

lowest : AV Superstop                  Best Food Mart / Des Moines   C Fresh Market                Cash Saver  /  E Euclid Ave   Cash Saver  /  Fleur         
highest: Walgreens #05852 / Des Moines Walgreens #07452 / Des Moines Walgreens #07453 / Des Moines Walgreens #07833 / Des Moines Walgreens #07968 / Des Moines
\end{verbatim}
}
\smallskip\hrule\smallskip
\noindent\textbf{Pack}\setlength{\unitlength}{0.001in}\hfill\begin{picture}(1.5,.1)(1500,0)\linethickness{0.6pt}
\put(0,0){\line(0,1){1}}
\put(67,0){\line(0,1){4}}
\put(89,0){\line(0,1){2}}
\put(111,0){\line(0,1){1}}
\put(134,0){\line(0,1){52}}
\put(178,0){\line(0,1){5}}
\put(200,0){\line(0,1){2}}
\put(223,0){\line(0,1){1}}
\put(245,0){\line(0,1){8}}
\put(267,0){\line(0,1){24}}
\put(290,0){\line(0,1){14}}
\put(312,0){\line(0,1){19}}
\put(334,0){\line(0,1){14}}
\put(356,0){\line(0,1){16}}
\put(379,0){\line(0,1){12}}
\put(401,0){\line(0,1){100}}
\put(423,0){\line(0,1){6}}
\put(445,0){\line(0,1){15}}
\put(468,0){\line(0,1){9}}
\put(490,0){\line(0,1){11}}
\put(512,0){\line(0,1){9}}
\put(535,0){\line(0,1){9}}
\put(557,0){\line(0,1){9}}
\put(579,0){\line(0,1){16}}
\put(601,0){\line(0,1){6}}
\put(624,0){\line(0,1){6}}
\put(646,0){\line(0,1){2}}
\put(668,0){\line(0,1){19}}
\put(690,0){\line(0,1){5}}
\put(713,0){\line(0,1){2}}
\put(735,0){\line(0,1){2}}
\put(757,0){\line(0,1){9}}
\put(780,0){\line(0,1){4}}
\put(802,0){\line(0,1){4}}
\put(824,0){\line(0,1){6}}
\put(846,0){\line(0,1){4}}
\put(869,0){\line(0,1){1}}
\put(891,0){\line(0,1){1}}
\put(935,0){\line(0,1){16}}
\put(1002,0){\line(0,1){2}}
\put(1025,0){\line(0,1){1}}
\put(1069,0){\line(0,1){2}}
\put(1091,0){\line(0,1){1}}
\put(1114,0){\line(0,1){4}}
\put(1136,0){\line(0,1){1}}
\put(1203,0){\line(0,1){1}}
\put(1292,0){\line(0,1){1}}
\put(1314,0){\line(0,1){1}}
\put(1381,0){\line(0,1){1}}
\put(1470,0){\line(0,1){4}}
\end{picture}

{\smaller
\begin{tabular}{ rrrrrrrrrrrrr }
n&missing&distinct&Info&Mean&Gmd&.05&.10&.25&.50&.75&.90&.95 \\
376&0&141&0.99&13.38&6.201& 6.00& 6.00& 9.75&12.00&16.00&21.37&24.00 \end{tabular}
\begin{verbatim}

lowest :  3.00000  4.50000  5.00000  5.25000  5.40000, highest: 30.00000 32.00000 32.57143 33.81818 36.00000
\end{verbatim}
}
\smallskip\hrule\smallskip
\noindent\textbf{Bottles.Sold}\setlength{\unitlength}{0.001in}\hfill\begin{picture}(1.5,.1)(1500,0)\linethickness{0.6pt}
\put(0,0){\line(0,1){100}}
\put(22,0){\line(0,1){46}}
\put(44,0){\line(0,1){23}}
\put(66,0){\line(0,1){15}}
\put(88,0){\line(0,1){7}}
\put(111,0){\line(0,1){3}}
\put(133,0){\line(0,1){2}}
\put(155,0){\line(0,1){3}}
\put(177,0){\line(0,1){2}}
\put(199,0){\line(0,1){2}}
\put(221,0){\line(0,1){1}}
\put(265,0){\line(0,1){1}}
\put(288,0){\line(0,1){1}}
\put(310,0){\line(0,1){1}}
\put(332,0){\line(0,1){1}}
\put(354,0){\line(0,1){1}}
\put(376,0){\line(0,1){1}}
\put(420,0){\line(0,1){1}}
\put(487,0){\line(0,1){1}}
\put(641,0){\line(0,1){1}}
\put(752,0){\line(0,1){1}}
\put(862,0){\line(0,1){1}}
\put(1084,0){\line(0,1){1}}
\put(1438,0){\line(0,1){1}}
\end{picture}

{\smaller[2]
\begin{tabular}{ rrrrrrrrrrrrr }
n&missing&distinct&Info&Mean&Gmd&.05&.10&.25&.50&.75&.90&.95 \\
376&0&153&0.999&116.1&173.3&  3.00&  5.00&  9.75& 32.00& 96.50&247.00&444.25 \end{tabular}
\begin{verbatim}

lowest :    1    2    3    4    5, highest: 1450 1704 1961 2431 3228
\end{verbatim}
}
\smallskip\hrule\smallskip
\noindent\textbf{Sale..Dollars.}\setlength{\unitlength}{0.001in}\hfill\begin{picture}(1.5,.1)(1500,0)\linethickness{0.6pt}
\put(0,0){\line(0,1){100}}
\put(20,0){\line(0,1){47}}
\put(40,0){\line(0,1){12}}
\put(60,0){\line(0,1){5}}
\put(80,0){\line(0,1){4}}
\put(101,0){\line(0,1){1}}
\put(121,0){\line(0,1){1}}
\put(141,0){\line(0,1){1}}
\put(161,0){\line(0,1){1}}
\put(201,0){\line(0,1){1}}
\put(221,0){\line(0,1){1}}
\put(241,0){\line(0,1){1}}
\put(282,0){\line(0,1){1}}
\put(523,0){\line(0,1){1}}
\put(543,0){\line(0,1){1}}
\put(604,0){\line(0,1){1}}
\put(724,0){\line(0,1){1}}
\put(785,0){\line(0,1){1}}
\put(805,0){\line(0,1){1}}
\put(1107,0){\line(0,1){1}}
\put(1429,0){\line(0,1){1}}
\end{picture}

{\smaller[2]
\begin{tabular}{ rrrrrrrrrrrrr }
n&missing&distinct&Info&Mean&Gmd&.05&.10&.25&.50&.75&.90&.95 \\
376&0&343&1&1891&3024&  44.98&  69.84& 162.63& 371.25&1078.08&2937.44&6963.48 \end{tabular}
\begin{verbatim}

lowest :     7.20    21.74    22.49    26.25    27.14, highest: 35818.26 38945.46 40135.98 54923.22 71157.84
\end{verbatim}
}
\smallskip\hrule\smallskip
\noindent\textbf{Bottle.Volume..ml.}\setlength{\unitlength}{0.001in}\hfill\begin{picture}(1.5,.1)(1500,0)\linethickness{0.6pt}
\put(0,0){\line(0,1){1}}
\put(76,0){\line(0,1){4}}
\put(95,0){\line(0,1){2}}
\put(113,0){\line(0,1){3}}
\put(151,0){\line(0,1){2}}
\put(170,0){\line(0,1){15}}
\put(189,0){\line(0,1){3}}
\put(208,0){\line(0,1){7}}
\put(227,0){\line(0,1){14}}
\put(246,0){\line(0,1){3}}
\put(265,0){\line(0,1){2}}
\put(284,0){\line(0,1){14}}
\put(302,0){\line(0,1){1}}
\put(321,0){\line(0,1){8}}
\put(340,0){\line(0,1){19}}
\put(359,0){\line(0,1){7}}
\put(378,0){\line(0,1){3}}
\put(397,0){\line(0,1){7}}
\put(416,0){\line(0,1){3}}
\put(435,0){\line(0,1){6}}
\put(454,0){\line(0,1){6}}
\put(473,0){\line(0,1){5}}
\put(492,0){\line(0,1){3}}
\put(510,0){\line(0,1){2}}
\put(529,0){\line(0,1){100}}
\put(548,0){\line(0,1){5}}
\put(567,0){\line(0,1){5}}
\put(586,0){\line(0,1){7}}
\put(605,0){\line(0,1){5}}
\put(624,0){\line(0,1){8}}
\put(643,0){\line(0,1){6}}
\put(662,0){\line(0,1){8}}
\put(681,0){\line(0,1){3}}
\put(699,0){\line(0,1){2}}
\put(718,0){\line(0,1){8}}
\put(737,0){\line(0,1){5}}
\put(756,0){\line(0,1){8}}
\put(775,0){\line(0,1){4}}
\put(794,0){\line(0,1){7}}
\put(813,0){\line(0,1){4}}
\put(832,0){\line(0,1){3}}
\put(851,0){\line(0,1){3}}
\put(870,0){\line(0,1){3}}
\put(889,0){\line(0,1){1}}
\put(907,0){\line(0,1){5}}
\put(926,0){\line(0,1){2}}
\put(945,0){\line(0,1){1}}
\put(964,0){\line(0,1){2}}
\put(983,0){\line(0,1){9}}
\put(1002,0){\line(0,1){2}}
\put(1021,0){\line(0,1){1}}
\put(1040,0){\line(0,1){1}}
\put(1059,0){\line(0,1){1}}
\put(1096,0){\line(0,1){2}}
\put(1153,0){\line(0,1){1}}
\put(1229,0){\line(0,1){2}}
\put(1286,0){\line(0,1){1}}
\put(1323,0){\line(0,1){1}}
\put(1475,0){\line(0,1){19}}
\end{picture}

{\smaller[2]
\begin{tabular}{ rrrrrrrrrrrrr }
n&missing&distinct&Info&Mean&Gmd&.05&.10&.25&.50&.75&.90&.95 \\
376&0&180&0.986&803.8&345.4& 375.0& 435.4& 573.8& 750.0& 953.8&1250.0&1564.3 \end{tabular}
\begin{verbatim}

lowest :  200.000  270.000  287.500  300.000  310.000, highest: 1416.667 1500.000 1550.000 1607.143 1750.000
\end{verbatim}
}
\smallskip\hrule\smallskip
\noindent\textbf{State.Bottle.Cost}\setlength{\unitlength}{0.001in}\hfill\begin{picture}(1.5,.1)(1500,0)\linethickness{0.6pt}
\put(0,0){\line(0,1){32}}
\put(18,0){\line(0,1){100}}
\put(35,0){\line(0,1){57}}
\put(53,0){\line(0,1){39}}
\put(71,0){\line(0,1){25}}
\put(89,0){\line(0,1){21}}
\put(106,0){\line(0,1){10}}
\put(124,0){\line(0,1){6}}
\put(142,0){\line(0,1){11}}
\put(160,0){\line(0,1){7}}
\put(177,0){\line(0,1){9}}
\put(195,0){\line(0,1){4}}
\put(213,0){\line(0,1){6}}
\put(231,0){\line(0,1){3}}
\put(248,0){\line(0,1){2}}
\put(266,0){\line(0,1){2}}
\put(284,0){\line(0,1){4}}
\put(319,0){\line(0,1){1}}
\put(355,0){\line(0,1){2}}
\put(390,0){\line(0,1){2}}
\put(426,0){\line(0,1){1}}
\put(514,0){\line(0,1){1}}
\put(532,0){\line(0,1){1}}
\put(568,0){\line(0,1){2}}
\put(621,0){\line(0,1){1}}
\put(639,0){\line(0,1){1}}
\put(710,0){\line(0,1){1}}
\put(745,0){\line(0,1){1}}
\put(780,0){\line(0,1){1}}
\put(816,0){\line(0,1){1}}
\put(922,0){\line(0,1){1}}
\put(1206,0){\line(0,1){1}}
\put(1455,0){\line(0,1){1}}
\end{picture}

{\smaller
\begin{verbatim}
       n  missing distinct     Info     Mean      Gmd      .05      .10      .25      .50 
     376        0      319        1    101.7    127.4    7.262   10.670   18.495   47.375 
     .75      .90      .95 
 103.537  225.940  338.075 

lowest :    3.21    3.46    3.50    4.10    4.40, highest:  887.95  918.77 1045.11 1362.12 1640.44
\end{verbatim}
}
\smallskip\hrule\smallskip
\noindent\textbf{State.Bottle.Retail}\setlength{\unitlength}{0.001in}\hfill\begin{picture}(1.5,.1)(1500,0)\linethickness{0.6pt}
\put(0,0){\line(0,1){16}}
\put(12,0){\line(0,1){100}}
\put(24,0){\line(0,1){66}}
\put(36,0){\line(0,1){42}}
\put(48,0){\line(0,1){45}}
\put(60,0){\line(0,1){29}}
\put(71,0){\line(0,1){23}}
\put(83,0){\line(0,1){14}}
\put(95,0){\line(0,1){14}}
\put(107,0){\line(0,1){11}}
\put(119,0){\line(0,1){6}}
\put(131,0){\line(0,1){5}}
\put(143,0){\line(0,1){11}}
\put(155,0){\line(0,1){5}}
\put(167,0){\line(0,1){8}}
\put(179,0){\line(0,1){7}}
\put(191,0){\line(0,1){2}}
\put(202,0){\line(0,1){6}}
\put(214,0){\line(0,1){5}}
\put(226,0){\line(0,1){1}}
\put(238,0){\line(0,1){4}}
\put(250,0){\line(0,1){1}}
\put(262,0){\line(0,1){4}}
\put(274,0){\line(0,1){1}}
\put(286,0){\line(0,1){2}}
\put(298,0){\line(0,1){1}}
\put(322,0){\line(0,1){1}}
\put(345,0){\line(0,1){1}}
\put(369,0){\line(0,1){1}}
\put(393,0){\line(0,1){2}}
\put(429,0){\line(0,1){1}}
\put(512,0){\line(0,1){1}}
\put(536,0){\line(0,1){1}}
\put(572,0){\line(0,1){2}}
\put(619,0){\line(0,1){1}}
\put(643,0){\line(0,1){1}}
\put(703,0){\line(0,1){1}}
\put(750,0){\line(0,1){1}}
\put(798,0){\line(0,1){1}}
\put(822,0){\line(0,1){1}}
\put(929,0){\line(0,1){1}}
\put(1215,0){\line(0,1){1}}
\put(1465,0){\line(0,1){1}}
\end{picture}

{\smaller[2]
\begin{tabular}{ rrrrrrrrrrrrr }
n&missing&distinct&Info&Mean&Gmd&.05&.10&.25&.50&.75&.90&.95 \\
376&0&322&1&152.6&191.3& 10.90& 16.01& 27.75& 71.09&155.34&338.93&507.33 \end{tabular}
\begin{verbatim}

lowest :    4.82    5.19    5.25    6.15    6.60, highest: 1332.12 1378.66 1568.12 2049.35 2467.91
\end{verbatim}
}
\smallskip\hrule\smallskip
\noindent\textbf{Volume.Sold..Gallons.}\setlength{\unitlength}{0.001in}\hfill\begin{picture}(1.5,.1)(1500,0)\linethickness{0.6pt}
\put(0,0){\line(0,1){100}}
\put(17,0){\line(0,1){45}}
\put(35,0){\line(0,1){20}}
\put(52,0){\line(0,1){10}}
\put(69,0){\line(0,1){4}}
\put(87,0){\line(0,1){5}}
\put(104,0){\line(0,1){2}}
\put(121,0){\line(0,1){3}}
\put(139,0){\line(0,1){2}}
\put(156,0){\line(0,1){1}}
\put(173,0){\line(0,1){1}}
\put(191,0){\line(0,1){1}}
\put(208,0){\line(0,1){1}}
\put(225,0){\line(0,1){2}}
\put(242,0){\line(0,1){1}}
\put(294,0){\line(0,1){1}}
\put(346,0){\line(0,1){1}}
\put(433,0){\line(0,1){1}}
\put(502,0){\line(0,1){1}}
\put(658,0){\line(0,1){1}}
\put(727,0){\line(0,1){1}}
\put(831,0){\line(0,1){1}}
\put(1074,0){\line(0,1){1}}
\put(1438,0){\line(0,1){1}}
\end{picture}

{\smaller[2]
\begin{tabular}{ rrrrrrrrrrrrr }
n&missing&distinct&Info&Mean&Gmd&.05&.10&.25&.50&.75&.90&.95 \\
376&0&254&1&24.4&38.04& 0.4225& 0.7350& 1.7900& 5.1500&18.2400&47.0350&88.4975 \end{tabular}
\begin{verbatim}

lowest :   0.13   0.20   0.30   0.32   0.39, highest: 383.96 417.74 483.05 623.52 830.55
\end{verbatim}
}
\smallskip\hrule\smallskip
}\end{spacing}

\begin{Shaded}
\begin{Highlighting}[]
\KeywordTok{describe}\NormalTok{(dfLiquorSales$Store.Number)}
\end{Highlighting}
\end{Shaded}

\begin{verbatim}
## dfLiquorSales$Store.Number 
##        n  missing distinct 
##      376        0       73 
## 
## lowest : 2190 2248 2527 2528 2532, highest: 5131 5132 5137 5145 5169
\end{verbatim}

\begin{Shaded}
\begin{Highlighting}[]
\KeywordTok{describe}\NormalTok{(dfLiquorSales$Category.Name)}
\end{Highlighting}
\end{Shaded}

\begin{verbatim}
## dfLiquorSales$Category.Name 
##        n  missing distinct 
##      376        0        8 
## 
## BLENDED WHISKIES (61, 0.162), CANADIAN WHISKIES (71, 0.189), IRISH
## WHISKIES (39, 0.104), SCOTCH WHISKIES (41, 0.109), SINGLE BARREL BOURBON
## WHISKIES (8, 0.021), STRAIGHT BOURBON WHISKIES (66, 0.176), STRAIGHT RYE
## WHISKIES (27, 0.072), TENNESSEE WHISKIES (63, 0.168)
\end{verbatim}

\section{Session Info}\label{session-info}

\begin{itemize}\raggedright
  \item R version 3.3.2 (2016-10-31), \verb|x86_64-w64-mingw32|
  \item Locale: \verb|LC_COLLATE=English_United States.1252|, \verb|LC_CTYPE=English_United States.1252|, \verb|LC_MONETARY=English_United States.1252|, \verb|LC_NUMERIC=C|, \verb|LC_TIME=English_United States.1252|
  \item Base packages: base, datasets, graphics, grDevices,
    methods, stats, utils
  \item Other packages: dplyr~0.5.0, fitdistrplus~1.0-7,
    Formula~1.2-1, ggplot2~2.2.0, Hmisc~4.0-1, lattice~0.20-34,
    logspline~2.1.9, MASS~7.3-45, pacman~0.4.1, pander~0.6.0,
    purrr~0.2.2, readr~1.0.0, stargazer~5.2, survival~2.40-1,
    tibble~1.2, tidyr~0.6.0, tidyverse~1.0.0
  \item Loaded via a namespace (and not attached): acepack~1.4.1,
    assertthat~0.1, backports~1.0.4, base64~2.0, cluster~2.0.5,
    colorspace~1.3-1, data.table~1.10.0, DBI~0.5-1, digest~0.6.10,
    evaluate~0.10, foreign~0.8-67, grid~3.3.2, gridExtra~2.2.1,
    gtable~0.2.0, htmlTable~1.7, htmltools~0.3.5, knitr~1.15.1,
    latticeExtra~0.6-28, lazyeval~0.2.0, magrittr~1.5,
    Matrix~1.2-7.1, munsell~0.4.3, nnet~7.3-12, openssl~0.9.5,
    plyr~1.8.4, R6~2.2.0, RColorBrewer~1.1-2, Rcpp~0.12.8,
    rmarkdown~1.2, rpart~4.1-10, rprojroot~1.1, rticles~0.2,
    scales~0.4.1, splines~3.3.2, stringi~1.1.2, stringr~1.1.0,
    tools~3.3.2, yaml~2.1.14
\end{itemize}

\section{R statistical programming
code.}\label{r-statistical-programming-code.}

Please see
\href{https://github.com/ChristopheHunt/DATA-621-Group-1/blob/master/Final\%20Project/Final\%20Project.Rmd}{Final
Project.rmd} on GitHub for source code.

https://github.com/ChristopheHunt/DATA-621-Group-1/blob/master/Final\%20Project/Final\%20Project.Rmd

\section*{References}\label{references}
\addcontentsline{toc}{section}{References}

\hypertarget{refs}{}
\hypertarget{ref-SpecialityGrow3}{}
Anonymous. 2016. ``Specialty Products Grow in Wine, Spirits.''
\emph{Beverage Industry} 107(7).

\hypertarget{ref-IowaSetsRecord2}{}
Boshart, Rod. 2001. ``Liquor Sales in Iowa Set Record.'' \emph{Gazette}.

\hypertarget{ref-KeepingSpiritsHigh1}{}
Del Buono, Amanda. 2016. ``Keeping Spirits High.'' \emph{Beverage
Industry} 107.4: 14--16, 18.

\end{document}


